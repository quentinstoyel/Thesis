
The objective of this work has been to develop a means to better simulate ELNES spectra in the context of lithium materials.  Lithium's sensitive nature has required the development of new experimental techniques which in turn have produced previously unattainable results requiring a more subtle approach than the literature standard.  The new method used to calculate core hole screening is particularly applicable to lithium, which is always sensitive to this effect, as seen with the limited amount of core hole screening in metallic lithium.  

The second main objective of this work was to highlight the need of further improvements to how ELNES spectra are approached in general.  The two largest areas of potential for theoretical calculations lie in their ability to quantify data and to explain and predict unknown results.  The current method in literature struggles to achieve either of these goals, and much of their success can be attributed to special case, as was demonstrated such as LiF.  In all of the other results, the inclusion of screening effects changed the analysis from being close enough to a far more exact agreement, quantifiable in the case of $ \mathrm{Li_2O}$ and predictive for the Li-LiF mixture.  

The method is currently limited to the lithium K edge, where core electron screening effects can safely be ignored.  However, as the results have shown, a more general screening approach is required for the full potential of ELNES simulations to be achieved.  .
