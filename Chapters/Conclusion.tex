
The main objectives of this work have been to develop a better means to simulate EELS spectra in the context of lithium materials and further verify the reliability of EELS at 30 keV.  The first order screening method presented in Chapter \ref{methods} is a step in the right direction.  It represents a dramatic improvement to the current methods and results in superior agreement between simulation and experiment in every case treated in this work.  Most importantly, the method achieves these improvements without requiring  empirical data or \textit{ad hoc} assumptions and maintains the computational scaling rate of previous methods. Ensuring that the method relies only on first principles enables it to reveal unintuitive results such as the unscreened hole in metallic lithium, and predict unknown systems such as the metallic lithium-lithium fluoride compound.  The success of the method at matching the experimental 30 keV EELS is a strong argument for both the validity and potential of low energy EELS as a tool to analyze beam sensitive materials.  \\

However, there is still considerable room for improvement in EELS calculations.  Experimental equipment and techniques continue to advance to study more intricate materials. The screening calculations in this work were tailored to lithium's various peculiarities and restricted to crystalline materials.  The improvements for lithium materials highlight the need for more general methods to support all elements, as well as amorphous and non infinite samples.  Additionally, the majority of theoretical EELS analysis remains qualitative and the realm of quantitative EELS presents an entirely new set of challenges. The ultimate goal of entirely predictive methods capable of performing quantitative analysis on any material remains out of reach and presents a target for future efforts. \\


 
