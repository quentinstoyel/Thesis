\documentclass[12pt]{report}
\usepackage{graphicx}
\graphicspath{{Images/}}
\usepackage{textcomp} %for the copywrite symbol
\linespread{1.6} %sets lines to 1.5 spacing.  1.6 would be double spaced
\usepackage[margin=1.0in]{geometry}
\usepackage[semicolon,round,sort&compress,sectionbib,numbers]{natbib}  
\usepackage{chapterbib}  
\usepackage{amsmath}
\usepackage{subcaption}
\usepackage{appendix} %for appendices



\usepackage{hyperref} %make stuff clickable
\hypersetup{
	colorlinks,
	citecolor=black,
	filecolor=black,
	linkcolor=black,
	urlcolor=black
}


\usepackage{physics}  %for bras and kets!
\newcommand{\angstrom}{\mbox{\normalfont\AA }} %makes \angstrom do its thing!

\begin{document}

Une nouvelle technique pour calcul\'ee l'intensit\'e de masquage des trous d'etas de c\oe ur est cr\'ee pour effectuer des simulations pr\'ecises de la spectroscopie de perte d'\'energie des \'electrons (spectroscopie EELS) des matireiux lithium. La methode est appliqu\'ee pour verifier la validit\'e de conduire la spectroscopie EELS \`a 30 keV pour minimizer les dommages de la faisceau d'\'electron. Des spectres EELS sont calcul\'ees pour $ \mathrm{Li_2O} $, LiF et le lithium metalique et une meilleur accordance avec les resultats exp\'erimentals est reconnue pour tous les mat\'erieux.  Des meillures caract\'eristiques quantitatives et pr\'edictives sont aussi obtenues.   La technique est bass\'ee sur la th\'eorie de la r\'eponse lin\'eaire et r\'elie la densit\'e des \'electrons aux effets de masquage.  Ces effets sont par la suite r\'epr\'esent\'es avec une trou non-enti\`ere dans des calculations th\'eorie de la fonctionnelle de la densit\'e. 



\end{document}