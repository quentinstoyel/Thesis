%Une nouvelle technique pour calcul\'ee l'intensit\'e de masquage des trous d'etas de c\oe ur est cr\'ee pour effectuer des simulations pr\'ecises de la spectroscopie de perte d'\'energie des \'electrons (spectroscopie EELS) des matireiux lithium. La methode est appliqu\'ee pour verifier la validit\'e de conduire la spectroscopie EELS \`a 30 keV pour minimizer les dommages de la faisceau d'\'electron. Des spectres EELS sont calcul\'ees pour $ \mathrm{Li_2O} $, LiF et le lithium metalique et une meilleur accordance avec les resultats exp\'erimentals est reconnue pour tous les mat\'erieux.  Des meillures caract\'eristiques quantitatives et pr\'edictives sont aussi obtenues.   La technique est bass\'ee sur la th\'eorie de la r\'eponse lin\'eaire et r\'elie la densit\'e des \'electrons aux effets de masquage.  Ces effets sont par la suite r\'epr\'esent\'es avec une trou non-enti\`ere dans des calculations th\'eorie de la fonctionnelle de la densit\'e. 


La spectroscopie par perte d'\'energie des \'electrons (EELS) est une technique de caract\'erisation sensible aux \'el\'ements de faible masse.  L'analyse EELS requiert des spectres de r\'ef\'erences pour valider les r\'esultats obtenus. Les simulations de spectres offrent une façon d'obtenir des r\'ef\'erences pour des nouveaux mat\'eriaux.  Bien que les simulations soient utiles, elles requi\`erent des ajustements de param\`etres.  Un de ces param\`etre est la fa\c{c}on de g\'erer un trou d'\'etat de c\oe ur. Une nouvelle technique pour calcul\'ee l'intensit\'e de masquage des trous d'\'electrons est propos\'ee, ce qui a pour effet d'augmenter la pr\'ecision des spectres EELS des mat\'eriaux contenant du lithium. La m\'ethode est appliqu\'ee pour v\'erifier la pertinence de conduire des exp\'eriences EELS \`a 30 keV, ce qui a pour effet de minimiser l'endommagement des \'echantillons induits par de le faisceau d'\'electron. La m\'ethode est bas\'ee sur la th\'eorie de la r\'eponse lin\'eaire, qui relie la densit\'e des \'electrons aux effets de masquage. Ces effets sont par la suite repr\'esent\'es par un trou partiel dans les calculs par la th\'eorie de la fonctionnelle de la densit\'e d'\'etats d'\'electrons. En utilisant la technique propos\'ee dans ce travaille, les spectres EELS calcul\'ees pour le $\mathrm{Li_2O}$, le LiF et le lithium m\'etallique sont en meilleur accordance avec les r\'esultats exp\'erimentaux de ces mat\'eriaux. Ces r\'esultats ouvrent la porte \`a des m\'ethode quantitatives bas\'ee sur la spectroscopie EELS.

